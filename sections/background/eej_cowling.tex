\subsection{赤道ジェット電流(EEJ)}
磁気赤道域で観測される地上磁場変動において、最も優勢な現象が赤道ジェット電流(Equatorial ElectroJet: EEJ)によるものである。
EEJは昼側の磁気赤道電離圏を東向きに流れる電流で、それによるH成分の磁場変動は最大で$+200$\,nT程度にもなる。
一方、西向きのジェット電流によってH成分が減少する変動も見られる。
これは赤道カウンタージェット電流(Equatorial Counter ElectroJet: CEJ)と呼ばれている。
CEJによる磁場変動は明け方や夕方に見られることが多い。

磁気赤道域電離圏特有のカウリング効果(Cowling Effect)がある。
これは、赤道域では地球の磁力線が地平に対して水平になることから生まれる効果である。
カウリング効果は磁気赤道を中心に非常に狭い緯度範囲でしか効かないため、EEJやCEJが流れるのも磁気緯度で$\pm 3$度以内の範囲だと言われている。

EEJやCEJが流れるためには電離圏電気伝導度が高いことに加え、外部から電場が印加される必要がある。
太陽風と磁気圏の相互作用で生まれる電場(太陽風中電場)の他にも、電離圏のプラズマと中性大気の相互作用によって大気が太陽によって温められて膨張し、最も温められる赤道正午付近から放射状に流れ出す際、プラズマを引きずりながら磁力線の間を移動することによって生じる。
荷電粒子が磁場の中を移動することで起電力が生じ、この起電力によって北半球では反時計回り、南半球では時計回りの電流系(Sq電流系)が形成される。
Sq電流系は中緯度での磁場の日変化パターンを作る主要な要素である。
ダイナモ電場によって朝側にはプラス、夕側にはマイナスのチャージが溜まる。このチャージによって東向きの電場が磁気赤道域に印加される。
EEJやCEJの変動を注視することで、磁気赤道電離圏や電離圏・大気圏相互作用の状態を監視することが可能である。
