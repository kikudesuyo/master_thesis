\subsection{MAGDAS観測網の概要}
世界最大の地上多点磁場観測ネットワークであるMAGDAS(図\ref{fig:magdas_network})は、磁気赤道上および210$^\circ$の磁気子午線上に沿って、南北半球を網羅するように、磁力計とそのデータ収集システムを密に設置しており、地磁気主磁場と作用して引き起こされるダイナモ作用に起因するSq場など全球的な磁場現象の緯度構造やその長期的・短期的変動を調べることができる。

九州大学の国際宇宙惑星環境研究センターが中心となって運用する地磁気観測ネットワークである。

\begin{itemize}
    \item 世界中に地上磁力計を設置しており、磁場データはリアルタイムに送られている。
\end{itemize}

\begin{figure}[htbp]
    \centering
    \includegraphics[width=0.8\textwidth]{figures/magdas_network.png}
    \caption{MAGDAS観測網}
    \label{fig:magdas_network}
\end{figure}
