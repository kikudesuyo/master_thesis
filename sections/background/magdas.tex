\subsection{MAGDAS観測網の概要}
MAGDASとは、九州大学国際宇宙惑星環境研究センターを中心に運用されている、世界最大規模の地上多点地磁気観測ネットワークである。
磁気赤道および磁気子午線210$^\circ$に沿って南北半球に磁力計とデータ収集システムを高密度に配置し、観測された磁場データはリアルタイムで送信されている。
これにより、地磁気主磁場とダイナモ作用に起因するSq場などの全球的磁場現象について、その緯度構造や長期・短期変動を解析することが可能である(図\ref{fig:magdas_network})。


\begin{figure}[htbp]
    \centering
    \includegraphics[width=0.8\textwidth]{figures/magdas_network.png}
    \caption{MAGDAS観測網}
    \label{fig:magdas_network}
\end{figure}
