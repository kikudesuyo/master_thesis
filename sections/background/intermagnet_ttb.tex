%!TEX root = ../../main.tex
\subsection{INTERMAGNETおよびTTBデータ}

INTERMAGNET (International Real-time Magnetic Observatory Network) は,地球磁場の変動をリアルタイムかつ高精度に監視し,データを提供することを目的とした国際的な地磁気観測所ネットワークである.
1980年代後半に設立され,世界各国の研究機関や大学が参加している.
INTERMAGNETに参加する観測所 (IMO: International Magnetic Observatory) は,測定機器やデータ処理に関して厳格な技術標準を遵守しており,これによりデータの品質と互換性が保たれている.
観測データは,標準的なデータ交換フォーマットであるIAGA-2002形式などで記録され,地域のGIN (Geomagnetic Information Node) と呼ばれるデータセンターに集約された後,科学コミュニティや商業利用のために公開されている.

本研究において解析対象とするTTBデータは,INTERMAGNETの観測所であるブラジルのTatuoca観測所 (IAGAコード: TTB) で取得されたものである.(表\ref{tab:ttb_station})
この観測所は磁気赤道の非常に近くに位置しているため,昼側の磁気赤道上空の電離圏E層を流れる強力な電流系である赤道ジェット電流 (Equatorial Electrojet: EEJ) の影響を直接的に観測することができる.
そのため,TTBデータはEEJの強度変動やその構造を研究する上で極めて重要な指標となる.
INTERMAGNETの高品質なデータ標準により,微細な磁場変動の解析や長期的な傾向の把握が可能となっている.

\begin{table}[htbp]
  \centering
  \caption{Tatuoca (TTB) 観測所の詳細}
  \label{tab:ttb_station}
  \begin{tabular}{llrrr}
    \toprule
    Code & Name & GG.Lat & GG.Lon  & Dip.Lat \\
    \midrule
    TTB & Tatuoca & -1.23 & -48.51 & -0.72 \\
    \bottomrule
  \end{tabular}
\end{table}
