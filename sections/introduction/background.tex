\subsection{研究背景}

2008年、九州大学国際宇宙天気科学・教育センター(ICSWSE)は、赤道ジェット電流(Equatorial Electrojet: EEJ)をモニタリングするための指標として、MAGDAS/CPMN観測網のデータを用いたEE-indexを提案した \cite{Uozumi2008}。
EE-indexは、赤道域の磁場変動を全球的な変動成分と局所的な変動成分に分離し、様々な電磁気現象を定量的にリアルタイムで監視することを目的としている。

EE-indexは、EDst(Equatorial Disturbance storm time)指数とEUEL(EE-index UEL)指数の2つの部分から構成される。
EDstは、赤道域における全球的な磁場変動を表しており、主に磁気嵐時のリングカレントや磁気圏界面電流の影響おおよび磁気圏極域擾乱の一部を含む。
一方、EUELは観測点直上のEEJやSq電流による局所的な磁場変動を表す。

Fujimoto et al. (2016) \cite{Fujimoto2016} により、観測点数の増加や緯度補正の導入といった改良が加えられ、長期的な解析が可能となった。
本研究では、この改良されたEE-indexを用いて解析を行う。
EE-indexの具体的な定義や改良手法の詳細については、第3章で述べる。

