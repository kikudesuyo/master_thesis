\subsection{今後の課題}
2026年2月現在、本システムにはパフォーマンスおよびスケーラビリティ、開発プロセスの観点でいくつかの課題が残されている。
主な課題は以下の3点である。

\begin{enumerate}
    \item \textbf{EE-index算出処理の効率化} \\
    EE-indexの算出結果は入力データが同一であれば常に同じ値となる(冪等性を持つ)にもかかわらず、本システムではリクエストの度に毎回算出処理を行っている。これにより、レスポンスタイムが著しく遅延する原因となっているため、計算結果の再利用が求められる。

    \item \textbf{データ管理とキャッシュ機構の導入} \\
    現状ではデータベースマネジメントシステムを使用せず、ファイルシステム上のデータを直接読み込んで処理を行っている。また、頻繁にアクセスされるデータに対するキャッシュ機構も導入されていない。今後トラフィックが増加した場合、サーバー負荷の増大や応答速度の低下が懸念されるため、効率的なデータ管理基盤の構築が必要である。

    \item \textbf{CI/CD環境の整備} \\
    現状、オンプレミス環境へのデプロイ作業は手動で行われており、更新のたびに手間とリスクが伴う。システムの安定稼働と迅速な機能改善を実現するためには、コードの変更を検知し、自動でビルド・テスト・デプロイを行うCI/CD(継続的インテグレーション・継続的デリバリー)環境の構築が不可欠である。
\end{enumerate}
