\subsection{環境構築・ドキュメント}
\label{sec:system_setup}
本プロジェクトでは、macOS環境での開発を前提とし、以下の手順で開発環境を構築する。
なお、本節に記載する情報は2026年2月時点のものである。
macOS におけるパッケージ管理を簡便に行うため、各種開発ツールのインストールには Homebrew の利用を推奨する。

\subsubsection{必要なツールのインストール}
まず、以下の開発ツールを導入する。

\begin{itemize}
    \item \textbf{Homebrew}: macOSのパッケージ管理ツール。
    \item \textbf{Git}: ソースコード管理ツール。
    \item \textbf{Docker}: コンテナ仮想化プラットフォーム。
    \item \textbf{uv}: 高速なPythonパッケージマネージャー(バックエンド用)。
    \item \textbf{Node.js}: JavaScript実行環境(フロントエンド用)。
\end{itemize}

\subsubsection{プロジェクトのセットアップ}
リポジトリをクローンした後、プロジェクトルートで以下のコマンドを実行することで、初期セットアップが完了する。

\begin{lstlisting}
git clone https://github.com/kikudesuyo/magdas
cd magdas
make init
\end{lstlisting}

\texttt{make init}コマンドにより、バックエンドの依存ライブラリ(\texttt{uv sync})およびフロントエンドの依存ライブラリ(\texttt{npm install})が一括でインストールされる。

\subsubsection{データ管理}
解析用の観測データ(.mgdファイル等)はリポジトリには含めず、ローカル環境の\texttt{backend/Storage}ディレクトリに個別に配置する運用としている。
これにより、機密性の高いデータや容量の大きなデータをGit管理から除外しつつ、開発環境で本番同様の解析フローを再現することを可能にしている。

\subsubsection{アプリケーションの起動}
開発時のアプリケーション起動方法は、以下の2パターンを用意している。

\begin{enumerate}
    \item \textbf{Dockerでの一括起動(推奨)}: \\
    \texttt{make up}コマンドを実行することで、バックエンドとフロントエンドをコンテナとして一括起動する。実環境との差異が少なく、手軽に動作確認を行うことが可能である。
    コンテナ起動後は、以下のURLでアクセス可能となる。
    \begin{itemize}
        \item フロントエンド: \url{http://localhost:5173}
        \item バックエンドAPI: \url{http://localhost:8000}
    \end{itemize}

    \item \textbf{ローカルでの個別起動}: \\
    デバッグ等の目的で、各サービスを個別に起動する場合に使用する。

    \textbf{バックエンド起動}:
\begin{lstlisting}
make be-dev
\end{lstlisting}

    \textbf{フロントエンド起動}:
\begin{lstlisting}
make fe-dev
\end{lstlisting}
\end{enumerate}
