\subsection{技術スタック}
本システムの開発および運用環境として、以下の技術スタックを選定した。
システムはオンプレミス環境で動作し、研究室のサーバー内に蓄積されたデータを活用する構成となっている。

\begin{itemize}
    \item \textbf{バックエンド}: Python, FastAPI
    \item \textbf{フロントエンド}: TypeScript, React
    \item \textbf{データベース (データソース)}: 研究室サーバー上の.magファイル (独自フォーマット)
    \item \textbf{実行環境}: オンプレミスサーバー
\end{itemize}

バックエンドにPython (FastAPI) を採用した主な理由は、数値計算ライブラリが充実しており、システム開発と並行してデータ解析を効率的に行える点にある。
Pythonは科学技術計算の分野で広く利用されており、本研究においてもデータの一次処理や解析アルゴリズムの実装に適している。
また、Matplotlib等の描画ライブラリが豊富であるため、解析結果の可視化や検証も円滑に行うことができ、研究活動とシステム開発の相互運用性を高める上で最適な選択であるといえる。

フロントエンドには、モダンなUI構築と保守性を重視し、TypeScriptおよびReactを使用している。
また、データの永続化・管理手法としては、既存の研究室サーバーに蓄積されているMAGDAS観測網のバイナリデータ(.magファイル)を直接読み込み、APIを通じてフロントエンドへ提供する形態をとっている。
