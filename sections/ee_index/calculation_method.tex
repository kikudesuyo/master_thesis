\subsection{EE-indexの算出方法}

EE-indexの算出には、磁気赤道域に設置された観測点で得られた地磁気水平成分(H成分)のデータを用いる。
まず、各観測点$s$におけるH成分$B_{H,s}$について、1日間のデータの中央値を基準値(ゼロレベル)として定義し、その基準値からの偏差を$ER_s$とする。
$ER_s$は以下の式で表される。

\begin{equation}
  ER_s(t) = B_{H,s}(t) - \text{median}(B_{H,s})_{\text{1day}}
\end{equation}

ここで、$t$は時刻を表す。
次に、全球的な擾乱成分であるEDst(Equatorial Disturbance storm time)を算出する。
赤道域の夜側(一般に18:00 - 06:00 LT)では、電離圏の電気伝導度が昼側に比べて著しく低下するため、電離圏起源の局所的な電流系による磁場変動は無視できる程度となる。
したがって、夜側に位置する観測点の磁場変動は、主にリングカレントや磁気圏界面電流といった全球的な磁気圏起源の変動を反映していると考えられる。
そこで、夜側(18:00 - 06:00 LT)に位置する全観測点の$ER_s$の平均値をとり、これをEDstと定義する。

\begin{equation}
  EDst(t) = \frac{1}{N_{night}} \sum_{s \in \text{Night}} ER_s(t)
\end{equation}

ここで、$N_{night}$は当該時刻において夜側に位置する観測点の総数である。

最後に、各観測点の$ER_s$からこのEDstを差し引くことで、全球的な変動成分を除去し、局所的な変動成分(主にEEJやSq電流に起因するもの)を抽出する。

\begin{equation}
  \text{EUEL}_s(t) = ER_s(t) - EDst(t)
\end{equation}

これにより、観測点直上の電離圏電流に起因する磁場変動成分を定量的に評価することが可能となる。

