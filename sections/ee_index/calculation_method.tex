\subsection{EE-indexの算出方法}
EE-indexの算出には磁気赤道域観測点のH成分を使用する。
1日分のH成分データの中央値をゼロレベルと定め、変動値を取り出す。その変動値を$ER$として定義する。

\begin{equation}
  ER_s = B_H - (B_H \text{の1日分の中央値}) \quad (s: \text{観測点})
\end{equation}

夜間(18-06LT)に位置する観測点の$ER$の平均を取り、これを$EDst$と定義する。

\begin{equation}
  EDst = \text{夜側観測点における } ER_s \text{ の平均値}
\end{equation}

昼側含む全観測点の$ER$から$EDst$を引き正の部分を$EU$、負の部分を$EL$と定義する。

\begin{equation}
  EU_s, EL_s = ER_s - EDst
\end{equation}

(1) $\sim$ (3)の操作で得られた$EDst$はグローバルな擾乱成分を表している。
なぜなら、夜側の電離圏では電気伝導度が昼側に比べて極端に小さくなり、その結果ローカルな電離圏電流は流れにくくなって、赤道環電流などのようなグローバルな現象による変動がメインとなるからである。
こうして得られた$EDst$をベースラインとして、H成分の生データ($ER$)からローカルな擾乱成分($EU$, $EL$)を捉えることが可能になる。
