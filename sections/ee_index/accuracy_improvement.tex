\subsection{EE-index精度向上方法}

Uozumi et al. (2008) \cite{Uozumi2008} は、MAGDAS/CPMN観測網のリアルタイムデータを用いて、赤道ジェット電流(EEJ)の短期的および長期的な変動を監視するための新しい指標としてEE-indexを提案した。
この初期の手法では、磁気赤道域の夜側(18-06 LT)にある観測点の水平成分磁場変動の平均値をEDst(Equatorial Disturbance in Storm Time)として定義していた。
EDstはDst指数の代替として機能し、各観測点のH成分からEDstを差し引くことで、局所的なEEJ成分(EU)およびカウンターエレクトロジェット(CEJ)成分(EL)を抽出することが可能である。
しかし、初期の研究段階では利用可能な夜側観測点の数が限られており、EDstの算出精度が観測点配置の制約を受けるという課題があった。

これに対し、Fujimoto et al. (2016) \cite{Fujimoto2016} では、長期的なEEJ変動解析に耐えうるよう、EE-indexに更なる改良が加えられた。
主な改良点は、EDstの算出に使用する観測点数の増加と、緯度補正の導入である。

改良版では、全経度をカバーする観測点を使用することで、より高精度に全球的な変動を見積もることが可能となった。
また、太平洋地域など赤道直下に観測点が存在しない経度帯における精度向上のため、低緯度観測点(磁気緯度$\pm 25^\circ$以内)のデータを用いて赤道上の磁場強度を推定する手法が導入された。
具体的には、観測された水平成分磁場$H_{obs}$と磁気緯度$\Phi$を用いて、赤道上の磁場強度$H_{dip}$を以下の式で推定する。

\begin{equation}
    H_{dip} = \frac{H_{obs}}{\cos \Phi}
\end{equation}

この補正により、赤道直下に観測点がない経度でも、近隣の低緯度観測点からEDstの算出に必要なデータを補完することが可能となり、EDstの算出精度が向上した。
これらの改良により、EE-indexは地磁気静穏時だけでなく、磁気嵐時などの地磁気擾乱時においても、赤道域の磁場変動を定量的に評価できる指標となっている。
