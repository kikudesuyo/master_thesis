\subsection{EE-indexの定義}
EE-indexは、磁気赤道付近の短期的・長期的な変動をリアルタイムでモニターすることを目的として九州大学が提案した新しい地磁気指標である \cite{Uozumi2008}。
EE-indexの算出には磁気赤道域MAGDAS観測点のデータが使用される。
MAGDASには、観測したデータをリアルタイムで転送するシステムが備わっているため、磁気赤道域で観測されたデータをすぐに解析指数化し、その時々の宙空の状態を速報することが可能である。
EE-indexはEDstとEUELの2つの指数から構成される。
EDstはグローバルな擾乱成分でDstの代用として利用することが出来る。
EUELは特定の地点でのみ変動するローカルな擾乱成分を指す。
EUは東向きのローカルな等価電流、ELは西向きのローカルな電流による変動に相当する。
これらの指数を算出することで赤道域の磁気擾乱のスケールの定量化が可能になる。
